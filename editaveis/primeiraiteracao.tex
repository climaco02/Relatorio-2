\part{Primeira iteração}
\chapter[Primeira iteração]{Primeira iteração}

\section{Introdução}

A primeira iteração do projeto, dentro contexto da disciplina de Requisitos de Software, é a fase do projeto que diz respeito a implementação dos casos de uso, definidos juntamente com o cliente e a professora. Tomando como base o ICSM \cite{boehmincremental}, a primeira iteração de implementação do projeto ocorrerá na primeira de Desenvolvimento, juntamente com a segunda fase de Fundamentação \cite{boehmincremental}.

\section{Definição dos requisitos da primeira iteração}

Foram definidos juntamente com a professora no dia 10/11/2016 os casos de uso a serem implementados na primeira iteração do projeto. Feita isso, essa definição foi validada com a cliente no mesmo dia. São estes o CUC01, CUC03 e AUC04 e o pla. Suas descrições podem ser encontradas no documento de arquitetura em \ref{doc:arq} e suas especificações em \ref{doc:cuc01}, \ref{doc:cuc03} e \ref{doc:auc04}, respectivamente.

\section{Planejamento da segunda iteração}

Foram priorizados juntamente com o cliente no dia 14/11/2016 mais três casos de uso a serem implementados na segunda iteração do projeto. 

\subsection{Dificuldades}

