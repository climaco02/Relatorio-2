\part{Primeira iteração}
\chapter[Primeira iteração]{Primeira iteração}

\section{Introdução}

A primeira iteração do projeto, dentro contexto da disciplina de Requisitos de Software, é a fase do projeto que diz respeito a implementação dos casos de uso, definidos juntamente com o cliente e a professora. Tomando como base o ICSM \cite{boehmincremental}, a primeira iteração de implementação do projeto ocorrerá na primeira de Desenvolvimento, juntamente com a segunda fase de Fundamentação \cite{boehmincremental}.

\section{Definição dos requisitos da primeira iteração}

Foram definidos juntamente com a professora no dia 10/11/2016 os casos de uso a serem implementados na primeira iteração do projeto. Feita isso, essa definição foi validada com a cliente no mesmo dia. São estes o CUC01, CUC03 e AUC04. Suas descrições podem ser encontradas no documento de arquitetura em \ref{doc:arq} e suas especificações em \ref{doc:cuc01}, \ref{doc:cuc03} e \ref{doc:auc04}, respectivamente.

\section{Planejamento da segunda iteração}

Foram priorizados juntamente com o cliente no dia 14/11/2016 mais três casos de uso a serem implementados na segunda iteração do projeto. São estes o AUC01, AUC02, AUC03 e CUC02, os casos de uso restantes. Suas descrições podem ser encontradas no documento de arquitetura em \ref{doc:arq} e suas especificações em \ref{doc:auc01}, \ref{doc:auc02}, \ref{doc:auc03} e \ref{doc:cuc02}, respectivamente.

\subsection{Dificuldades}

Durante a priorização dos casos de uso, o cliente não se mostrou interessado em contribuir com o dedesenvolvimento do projeto devido sobrecarga com atividades da empresa. Esse problema foi resolvido priorizando os casos de uso com a professora, como sugere o plano de ensino e apenas validando os casos de uso com o cliente.

\section{Rebaseline dos casos de uso}

Após o desenvolvimento dos casos de uso, estes foram entregue ao cliente para serem validados. O cliente ficou satisfeito com a implementação dos casos de uso CUC01 e CUC03. No entanto, ele pediu para que o AUC04 fosse mudado, pois não gostou de ter que realizar login para fazer as atividades desejadas, como funcionava a regra de negócio do caso de uso. Vale ressaltar que esse incontentamento do cliente não foi identificado anteriormente por meio das técnicas de elucidação de requisitos.

Uma vez o cliente tendo feita essa exigência, o caso de uso foi refeito e a documentação atualizada. A documentação de todos os casos de uso que se baseavam na mesma regra de negocio também foi atualizada respeitando a decisão do cliente.
